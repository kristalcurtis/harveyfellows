\documentclass{article}
\usepackage{geometry}
\usepackage{nopageno}
\geometry{top=1.0in, bottom=1.0in, left=1.0in, right=1.0in}

\begin{document}

\pagestyle{plain}

\noindent\textbf{4.  Reference at least one recent leadership experience related to your vocational goal. If possible, show how you have demonstrated the ability to lead others, particularly as the head of an organization or group. Describe how this experience has played a role in your development and/or informed your understanding of leadership in general.}\\

Over the past year and a half, I have supervised an undergraduate research assistant, Durga Kandasamy.  
Though we have several lab-wide research assistants, I am one of the few graduate students to have a dedicated undergraduate.
Initially, she was unfamiliar with many tools needed for our research, since several are outside the scope of the undergraduate curriculum.
However, she was enthusiastic, so I gave her a chance and helped her learn about version control (github), dependency management (maven/sbt), data analysis (R), languages (Scala), and web (JavaScript, jQuery, HTML, REST).
I also taught her about several of our lab's research areas, including cloud computing, machine learning, and crowdsourcing, and genomics.
The most difficult part about managing Durga has been experimenting with finding the right tasks for her to work on.
At first, I would give her an assignment, and she would say she understood what I wanted, but then she would come back with something completely different.
I learned to ask her to repeat back to me what she understood from the assignment, and that allowed us to save a lot of time and frustration.

This experience has significantly impacted my views about leadership.
I have learned that when I deal with students, it helps to be specific since they are used to doing class assignments which are usually tightly scoped and clearly defined.
Instead of simply blaming my students, I need to be willing to acknowledge that I could have communicated more clearly.
It is important to remember that I have weaknesses too, and that I benefited from patient leadership when I was in their position.
I also enjoy helping people develop, whether in gaining familiarity with real-world technologies or practicing broader concepts like self-motivation, time management, and collaboration.
I love knowing that my work with students will help them prepare for the next phase of their lives (whether going on to graduate school or to work in industry).
It is really rewarding when I see someone starting to put into practice what I taught them; I have seen Durga grow over the time we have worked together.
She told me that because of her time working with me, she came to believe that she could actually succeed in graduate school, which was not something she thought before.
Now, I am preparing to bring on a new undergraduate, and I look forward to what we will teach each other.
\end{document}