\documentclass{article}
\usepackage{geometry}
\usepackage{nopageno}
\geometry{top=1.0in, bottom=1.0in, left=1.0in, right=1.0in}

\begin{document}

\pagestyle{plain}

\noindent\textbf{4.  Reference at least one recent leadership experience related to your vocational goal. If possible, show how you have demonstrated the ability to lead others, particularly as the head of an organization or group. Describe how this experience has played a role in your development and/or informed your understanding of leadership in general.}\\

Over the past year and a half, I have been supervising an undergraduate research assistant, Durga Kandasamy.  
A friend initially connected us.
My advisors gave me the authority to hire her to be my personal research assistant rather than a general lab assistant.  Though we do have several lab-wide research assistants, I am one of the few graduate students to have a dedicated undergraduate.
I hired her partially as an outreach, because I value giving opportunities in computing to women.
Initially, she was unfamiliar with many tools and technologies that she needed for the research we were doing.
However, I believed she had enthusiasm and potential to learn and grow, so I gave her a chance and helped her learn about a broad spectrum of things, including version control (github), dependency management (maven/sbt), data analysis (R), new programming languages (Scala), and web programming (JavaScript, jQuery, HTML, REST).
I also helped her learn about several research areas that our lab is involved with, including cloud computing, machine learning, and crowdsourcing; now, she is also getting familiar with genomics.
The most difficult part about managing Durga has been experimenting with finding the right tasks for her to work on.
At first, I would give her an assignment, and she would say she understood what I wanted, but then she would come back with something completely different from what I was expecting.
Rather than getting frustrated, I took the opportunity to work on my communication skills.
I would ask her to repeat back to me what she understood from the assignment, and that allowed us to save a lot of time and frustration.

Through this experience, I have learned a lot about leadership.
Sometimes, it is hard to get your point across; instead of just getting frustrated with the people you are supervising, however, you need to work harder to communicate clearly, being willing to acknowledge that you could have done a better job specifying the parameters of the task.
I have learned that when you are dealing with undergrads, you have to be very specific with what you want because they are used to doing class assignments which are always very tightly scoped and clearly defined.
An aspect of leadership that I like is that you get the opportunity to help people develop their skills and learn real-world technologies, as well as useful concepts like working independently, time management, and how to collaborate.
I love the feeling that what you teach someone will help them in whatever they do after they leave your leadership (whether going on to graduate school or going to work in industry).
Sometimes it is tempting to get frustrated, but it is important to remember that you have weaknesses too, and that you benefited from patient leadership when you were in their position.
It is really rewarding when you see someone starting to put into practice what you taught them; I have seen Durga grow over the time we have worked together in terms of her ability to work independently and efficiently.
She also told me that because of her time working with me, she came to believe that she could actually succeed in graduate school, which was not something she thought before working with me.
Now, I am preparing to bring on a new undergraduate, and I look forward to what we will teach each other.

\end{document}