\documentclass{article}
\usepackage{geometry}
\usepackage{nopageno}
\geometry{top=1.0in, bottom=1.0in, left=1.0in, right=1.0in}

\begin{document}
	
\pagestyle{plain}

\textbf{Write a personal statement of your Christian faith (what it is that you believe). Explain the significance of your faith in Jesus Christ to your everyday life. Be sure to describe your involvement in a local church, and give examples of opportunities that you have had to demonstrate and evidence your faith.}\\

As an evangelical Christian, I believe that the Bible is God's inspired word, in which He reveals His character and His purposes for the world\footnote{1 Tim. 3:16-17}.
According to the Bible, God is triune, that He is the only God, and that He created the universe.
It also teaches that Jesus is God's only Son, that He is fully God and fully human, and that he was incarnate to provide salvation for mankind.  
Jesus died on the cross, rose from the dead, ascended into heaven, and is seated at God's right hand, where he intercedes for us\footnote{Heb. 4:14-16, Romans 8:26}. 
He will return one day, and the saints will be given glorified bodies and ushered into God's presence for eternity.
If anyone has faith in Jesus's atoning sacrifice and asks Jesus to be his/her Savior and Lord, they will be saved\footnote{Rom. 10:9-10}.
Faith in Jesus as Savior and Lord is the only path to salvation; anyone who rejects Him will be destined for eternal separation from God\footnote{Acts 4:12}.  Thus, we who know Jesus are obligated to spread the gospel so that all might have an opportunity to know Him\footnote{2 Cor. 5:20, Rev. 5:9}.\footnote{My statements here are influenced by the Nicene Creed.}

God calls us to live out our faith in the context of a local church body, since on your own, it is too hard to take up your cross daily and too easy to fall into sin.
For me, this means membership in a church, where I attend Sunday services and belong to a community group.  
In community groups, we work out our salvation\footnote{Phil. 2:12} through Bible study and prayer.  We also serve together, both at our church and in the community.  This helps us avoid being just a social group; instead, we practice following God.
Over the past few years, I've volunteered alongside my group members in various ways, including doing yardwork for an elderly man, beautifying a local middle school, cleaning at a pregnancy center, and serving concessions at a carnival for graduate student families.  This year, we will be cooking meals for homeless youth.
I am committed to service because God commanded us to remember the least among us\footnote{Matt. 25:41-45}.

God has given me the opportunity to study at an elite institution, and I believe that He wants me to shine a light to my colleagues.  People come from all over the world to study here at Berkeley, so I can be a missionary without leaving home.  Thus, I strive to share my faith with my friends and colleagues through relational evangelism.
\end{document}