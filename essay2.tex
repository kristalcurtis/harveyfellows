\documentclass{article}
\usepackage{geometry}
\usepackage{nopageno}
\geometry{top=1.0in, bottom=1.0in, left=1.0in, right=1.0in}

\begin{document}

\textbf{Briefly outline your vocational goals. Describe the factors that have motivated you to pursue your vocation. Be sure to reflect on your efforts to integrate your faith, learning and vocation.}\\
	
\pagestyle{plain}

I aspire to have a career in which I do research, teach, and supervise students in a university setting.
My research interests are in applying statistics and distributed systems to the analysis of large-scale data.
In particular, I am interested in the area of genomic research for cancer treatment.  My goal is to innovate regarding the processing of genomic data in such a way that it makes a difference in how cancer patients receive treatment.  This is important because personalized treatment is not only more likely to be effective but also reduces the trauma of myriad rounds of chemotherapy and radiation.\footnote{http://www.genome.gov/13514107}

% Could omit this paragraph
%I was drawn to computer science (CS) beginning with my interest in math and science, which led me to attend a magnet high school focused on technology.
%After getting the opportunity to learn about electronics, I decided to pursue electrical engineering, but my undergraduate advisor suggested that I major in CS as well, and I found that I was much more enthusiastic about CS.
%Writing a computer program is an incredible creative process where it feels like you're making something from nothing: you start with the language and a blank file, and you end up with something useful.
%Also, since programs must be exactly right, excellence in attention to detail is essential.

Several factors have contributed to my interest in working as an academic.
My father encouraged me to pursue a doctorate, saying that research would be much more interesting than a regular programming job.
I was able to do research for several years in college, and I worked on a NASA-sponsored project whose goal was to develop technology to allow scientists to remotely conduct experiments.  I loved the challenges involved in tackling open problems.
Regarding teaching, serving as a TA for my advisor inspired me to teach students about current technologies that will help them in the workforce.
Finally, I had a graduate student mentor when I was an undergraduate research assistant, and she had a big impact on me.  She taught me a lot, encouraged me to grow as a student, and believed I would succeed.

I believe God has called me to integrate my faith with my vocation, and working as a CS academic provides a lot of opportunities to do so.
Research in CS lets me exercise my God-given creativity, and the genomics application allows me to not only learn more about God's creation but also to work on a project that has the potential to positively impact society by bringing shalom and healing.
Regarding teaching and supervising students, I leverage my knowledge of my own weaknesses to be patient and compassionate to students when they are struggling.\footnote{2 Cor. 12:9-10}
I strive to help students position themselves so that they can succeed in whatever comes next in their career trajectories.
The CS community is very international and has very few Christians.
Thus, I strive to witness through sharing the gospel verbally and by my example.
I have a heart for missions but do not feel called to serve overseas, and this gives me a way to serve as a missionary to under-reached people groups while staying within the country.

\end{document}
