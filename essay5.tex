\documentclass{article}
\usepackage{geometry}
\usepackage{nopageno}
\geometry{top=1.0in, bottom=1.0in, left=1.0in, right=1.0in}

\begin{document}

\pagestyle{plain}

\noindent\textbf{5.  Describe the special strengths of the degree program(s) to which you have applied, and how they compare to the acknowledged premier programs in your specialty. Discuss how these programs will prepare you to enter the job market and pursue a position of influence within your field.}\\

The computer science program at Berkeley ties for first place with Stanford, MIT, and Carnegie Mellon\footnote{http://grad-schools.usnews.rankingsandreviews.com/best-graduate-schools/top-science-schools/computer-science-rankings}.  
Stanford tends to have more of an entrepreneurial focus, since it is located close to Silicon Valley.  
They have a large master's program, where people usually pay to get a degree comprised mostly of coursework.
The graduate program at Berkeley is more focused on research.  We have relatively few master's students; most graduate students are pursuing a PhD.
We also benefit from being near Silicon Valley, however; we have lots of opportunities to give talks to companies, and they often visit us to give talks as well.
Carnegie Mellon and MIT are predominantly technical universities (their strongest programs are in engineering, computer science, or the physical sciences), whereas Berkeley is an all-around top institution.  
Therefore, we are in a better position to do interdisciplinary work, since it is easy to find excellent collaborators in other departments.
The AMP Lab, which I work in, leverages the availability of collaborators in other areas to find applications to drive our research in big data.

Getting a PhD from Berkeley will give me strong preparation for entering the job market.
Coming from a top program gives me instant credibility.
I have the opportunity to learn from and work with amazing faculty who have made some seminal achievements in CS (e.g., Inktomi, an influential early search engine; RAID, a technology for utilizing multiple hard drives, RISC, a simplified instruction set that enables huge efficiency gains).
I took my theory course from a Turing Award winner (highest honor in CS), and I'm currently collaborating with him on a genomics project.
My fellow students are also incredible; they're very helpful and therefore make great collaborators.
They're also some of the smartest people I've ever met, and I've learned a ton from them.

Regarding the AMP Lab in particular, we have lots of faculty and students from various subdisciplines in CS (e.g., machine learning, networks, systems, databases).
Therefore, we have lots of opportunities for interesting collaborations wherein we can leverage the diverse skill sets.
Our lab has a tradition of taking semi-annual retreats with our industrial sponsors, so we have lots of opportunities to present our work in a supportive environment.
This is great training for presenting my work at conferences.
We also have invaluable networking opportunities with folks in academia (visitors to the lab) and industry (attendees at the retreats) which will help us to find jobs when we graduate.

Getting a PhD from Berkeley will help me to be a person of influence within the field.
I hope to have influence through the research I do and the people I work with.
I am learning to do research from a famous advisor, David Patterson, who is focused more on impact than sheer number of publications.
He encourages me to pursue problems that will make a big impact in our discipline and in the sciences (the cancer application).
Another feature of our lab is that we actually turn our research into working artifacts.
Since our lab is so collegial, I learn a ton from my colleagues, and that helps me grow as a researcher.

Both of my advisors enjoy teaching and working with students, so I get to learn a lot.
My advisors allowed me to hire two undergraduate research assistants.
Thus, I'm learning how to be a manager.
I had the opportunity to serve as a TA twice, with two different professors.
When I was a TA with one of my advisors, Armando Fox, I got to practice some higher-level course organization tasks like figuring out the course schedule, designing assignments, and setting policies.

Overall, my education at Berkeley is helping me grow as a researcher, manager, and teacher, and I believe this will help me to be successful and influential in my career.

\end{document}