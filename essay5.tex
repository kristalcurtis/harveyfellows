\documentclass{article}
\usepackage{geometry}
\usepackage{nopageno}
\geometry{top=1.0in, bottom=1.0in, left=1.0in, right=1.0in}

\begin{document}

\pagestyle{plain}

\noindent\textbf{5.  Describe the special strengths of the degree program(s) to which you have applied, and how they compare to the acknowledged premier programs in your specialty. Discuss how these programs will prepare you to enter the job market and pursue a position of influence within your field.}\\

Berkeley is an amazing place to pursue a PhD.
According to the US News rankings from 2010, our computer science (CS) graduate program ties for first place with Stanford, MIT, and Carnegie Mellon\footnote{http://grad-schools.usnews.rankingsandreviews.com/best-graduate-schools/top-science-schools/computer-science-rankings}.  
% sentence about graduates' job success
Our PhD graduates go on to work as postdocs and faculty members at top universities, and they have tremendous success at key technology companies.
Berkeley also has immense strengths outside CS.
According to a 2011 National Research Council report, 48 out of 52 Berkeley PhD programs are in the top ten.
Therefore, we produce high-quality interdisciplinary work.
The AMP Lab, where I work, leverages the availability of excellent collaborators in other areas to find big data applications.

Getting a PhD from Berkeley will give me strong preparation for entering the job market, largely because of the people I encounter.
My main advisor, David Patterson, has made seminal achievements in CS (e.g., RAID, a technology for utilizing multiple hard drives; RISC, a simplified processor instruction set that enables huge efficiency gains), and I am currently collaborating with a Turing Award winner (highest honor in CS), Richard Karp, on a genomics project.
My fellow students are incredible, and I have learned a ton from them.
In the AMP Lab, we have faculty and students from various subdisciplines in CS (e.g., machine learning, networks, systems, databases).
This diversity helps us find research opportunities that others miss.
Our lab has a tradition of taking semi-annual retreats with our industrial sponsors, so we get regular practice presenting our work in a supportive environment.
This is great training for presenting my work at conferences.
We also have invaluable networking opportunities with folks in academia (visitors to the lab) and industry (attendees at the retreats) which will help us to find jobs when we graduate.

My education will enable me to be a person of influence within the field.
My advisors encourage me to pursue problems that impact our discipline and the sciences.
In addition, they enjoy supervising and teaching students.
Rather than resenting these duties, they make a serious effort to excel, and they have given me opportunities to build these skills.
My advisors allowed me to hire two undergraduate research assistants.
I had the opportunity to serve as a TA twice, with two different professors.
Berkeley is helping me grow as a researcher, manager, and teacher, and I believe this will help me to be successful in my career.

\end{document}