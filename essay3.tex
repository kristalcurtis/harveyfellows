\documentclass{article}
\usepackage{geometry}
\usepackage{nopageno}
\geometry{top=1.0in, bottom=1.0in, left=1.0in, right=1.0in}

\begin{document}

\pagestyle{plain}

\noindent\textbf{3.  Explain how this vocational arena impacts society at large, and how this vocation is strategic to Christ�s kingdom.  Provide evidence that your chosen field is under-represented by Christians and tell us how you plan to impact your field for Christ.}\\

The field of computing has a significant impact due to its ubiquity.  
As of 2010, there were more than five billion mobile phones\footnote{http://www.bbc.co.uk/news/10569081} and at least two billion internet connections\footnote{http://www.engadget.com/2011/01/28/un-worldwide-internet-users-hit-two-billion-cellphone-subscript/}.
Thus, people all over the world have unprecedented access to all sorts of information.
They also produce huge volumes of data, whether directly (e.g., blog posts) or indirectly (e.g., GPS data).
Thus, data analysis has become one of the biggest challenges in computing.

The AMP Lab at Berkeley is focused on how you can use \textbf{A}lgorithms, \textbf{M}achines, and \textbf{P}eople to make sense of big data.  
Big data is challenging to tackle because it is often heterogeneous, uncurated, inconsistent, and rapidly accumulating.
In addition, many of the questions people would like ask of big data are time-sensitive.
Our research agenda is shaped by applications that produce and analyze big data.
I am focusing on personalized medicine, particularly the problem of recommending cancer treatments based on a patient's normal and cancerous genomic information.
%Several large cancer datasets exist\footnote{http://cancergenome.nih.gov/, www.1000genomes.org/}; however, this data is not yet being used to help physicians reach decisions about their patients' treatment.
A current hurdle is that you need to sequence a tumor regularly to track its progress.
If this is to be feasible, sequencing must be affordable.
The cost of obtaining the sequence data is rapidly decreasing\footnote{http://www.genome.gov/sequencingcosts/}; soon, the dominating factor will be the data processing.
To address this, my colleagues and I are working on a new algorithm that will quickly, cheaply, and accurately process genome data.

Computing has a global reach.
People from under-reached areas (such as India and China) not only consume computing but also make up a significant portion of its workforce.
This allows us to impact people even in areas that are hostile to Christians.
In addition, computing has the potential to bring shalom to our society.
Personalized medicine is an example; in addition, because of the internet, people in rural areas have access to a plethora of information that can aid in their development.

Though computing is a strategic field to reach the world for Christ, Christians are severely under-represented.
By one figure, only 23\% of computer science undergraduate students have a spiritual life of any kind\footnote{Lindholm, Jennifer A.  "The 'Interior' Lives of American College Students."  \textit{Passing on the Faith -- Transforming Traditions for the Next Generation of Jews, Christians, and Muslims.}  Ed. James L. Heft.  New York:  Fordham University Press, 2006.  75-102.}.

I believe God has called me to work in the field of computing so that He might use me to influence it for Him.
Thus, I strive to work on problems like the cancer application that can benefit society.
I also work to reach the global lost without leaving the US by sharing the gospel through word and deed with my colleagues.

\end{document}