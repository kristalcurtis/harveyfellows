\documentclass{article}
\usepackage{geometry}
\usepackage{nopageno}
\geometry{top=1.0in, bottom=1.0in, left=1.0in, right=1.0in}

\begin{document}

\pagestyle{plain}

\noindent\textbf{6.  Please state anything else that is important for us to know about you. We encourage you to share information you feel is important, but have not had the opportunity to fully present in the other essays. You may wish to explain gaps in work/study in your CV, explain your work sample, expound upon your vocational goals, give another example of your leadership gifts, etc.}\\

It can be difficult to navigate a career in technology as a woman.
First, you often feel isolated.
As a PhD student, once you finish classes, you rarely encounter people outside your male-dominated group.
It can be unnerving to be the only woman in a room; you may question whether your presence is warranted.
There is a strong subculture of sci-fi and video games, and other interests may be ridiculed.
Another hurdle is fearing that women lack the abilities to succeed in technology.
Many men are aggressive, often interrupting or talking over each other, so you can feel that you have to be impolite to be heard.
Also, while men are usually seen as focused, women are considered social, which can lead to wondering if they are well-suited for CS, which requires extreme attention to detail.
Women may also doubt that they can attain both professional and personal success.
Given the time investment required to be an academic, many women delay having children until it becomes difficult.
Since there are very few female professors in CS, it can be challenging to find a role model who has managed to have a family.

I have found that relationships with other women help me in this struggle.
In college, I worked with a graduate student who I admired, and her encouragement had a big impact on me.
When I arrived at Cal, I joined WICSE, a group for graduate women in computing.
Since it involves people from many research groups, I see lots of examples of successful female graduate students.
I have a close female friend in my lab, and we have several shared interests, which helps my sense of belonging.
I also attend Grace Hopper, an annual conference for women in computing where women are candid about their work/life balance.
Hearing people's stories gives me ideas about how I can manage my own career and family.

Due to my gratitude for the support I have received, I plan to make mentoring a key part of my career.
My research assistant is interested in graduate school, and I am shepherding her through the application process.
I served as an officer in WICSE for two years because of my appreciation for the organization.
I also regularly serve on panels regarding success in graduate school.
Technology is an amazing field, and any woman who wants to be involved should be empowered to give it a shot!

\end{document}