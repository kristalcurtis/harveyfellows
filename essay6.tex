\documentclass{article}
\usepackage{geometry}
\usepackage{nopageno}
\geometry{top=1.0in, bottom=1.0in, left=1.0in, right=1.0in}

\begin{document}

\pagestyle{plain}

\noindent\textbf{6.  Please state anything else that is important for us to know about you. We encourage you to share information you feel is important, but have not had the opportunity to fully present in the other essays. You may wish to explain gaps in work/study in your CV, explain your work sample, expound upon your vocational goals, give another example of your leadership gifts, etc.}

It is difficult to be a woman in technology.
The first hurdle is feeling isolated.
It can hurt your confidence when you are the only woman in a room.
It can make you question why you are there, whether something is wrong with you, and whether women are supposed to be there at all.
You can feel like an outsider because you may not be as involved with the nerd subculture (e.g. Star Wars, video games) as your colleagues, while your interests (e.g. fashion) may be ridiculed or trivialized.
Once you finish classes, you really only see the women in your lab or group, which can be very few.

The second hurdle is believing that women don't have the abilities to succeed in technology.
Men are very aggressive, often interrupting or talking over each other to get a word in, so you feel like you can't act as you normally would and still manage to be heard.
Men are usually seen as being more focused, whereas women are seen as being more social, so you can wonder if women are well-suited for research, especially when laster focus, extreme attention to detail, and dedication are required to do a PhD in CS.

The third hurdle is believing that women can't succeed in technology and still have a family life.
You hear a lot of horror stories about the time investment required to be an academic, because of the tenure clock, and how many women delay having children till they are too old to conceive.
There aren't a lot of role models demonstrating that professional and familial success is possible for women.
There are very few female professors, and they often don't have children.

One of the ways I deal with these obstacles is through mentoring.
When I was an undergraduate, my graduate student supervisor believed in me and encouraged me to develop good habits.
When I arrived at Cal, I joined WICSE, our group for graduate women in computing.
Since it involves people from many research groups besides my own, this lets me see lots of examples of women who succeed in graduate school.
I have a close female friend in my lab, and we attend a lot of the same meetings, so I am usually not the only woman in the room.
In addition, we have a lot of shared interests, so I feel like people like me do belong in the field.
I also attend Grace Hopper, an annual conference for women in computing.
There, lots of women are willing to be very candid about their families and work/life balance.
Hearing people's stories gives you ideas about how you could make it work in your own life and also about things you want to avoid doing.

Because mentoring has been very important to me, I am also making an effort to mentor others.
I make a point of giving undergraduate research opportunities to women.
My current undergraduate research assistant is interested in graduate school, and I'm shepherding her through the process.
I served as an officer in WICSE for two years because of how much I appreciate and believe in the organization.
At our monthly mentoring lunch, I encourage undergraduates to get involved research.
I also regularly serve on various panels regarding how to apply for graduate school or fellowships or general advice about succeeding in graduate school.

Making it as a woman in technology is very difficult, but I've been fortunate to get a lot of support from mentoring, and I plan to make mentoring a big part of my career as well.
Technology is a great field, and any woman who wants to be involved should be empowered to give it a shot!

\end{document}